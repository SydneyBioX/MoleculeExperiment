% Options for packages loaded elsewhere
\PassOptionsToPackage{unicode}{hyperref}
\PassOptionsToPackage{hyphens}{url}
%
\documentclass[
]{article}
\usepackage{lmodern}
\usepackage{amssymb,amsmath}
\usepackage{ifxetex,ifluatex}
\ifnum 0\ifxetex 1\fi\ifluatex 1\fi=0 % if pdftex
  \usepackage[T1]{fontenc}
  \usepackage[utf8]{inputenc}
  \usepackage{textcomp} % provide euro and other symbols
\else % if luatex or xetex
  \usepackage{unicode-math}
  \defaultfontfeatures{Scale=MatchLowercase}
  \defaultfontfeatures[\rmfamily]{Ligatures=TeX,Scale=1}
\fi
% Use upquote if available, for straight quotes in verbatim environments
\IfFileExists{upquote.sty}{\usepackage{upquote}}{}
\IfFileExists{microtype.sty}{% use microtype if available
  \usepackage[]{microtype}
  \UseMicrotypeSet[protrusion]{basicmath} % disable protrusion for tt fonts
}{}
\makeatletter
\@ifundefined{KOMAClassName}{% if non-KOMA class
  \IfFileExists{parskip.sty}{%
    \usepackage{parskip}
  }{% else
    \setlength{\parindent}{0pt}
    \setlength{\parskip}{6pt plus 2pt minus 1pt}}
}{% if KOMA class
  \KOMAoptions{parskip=half}}
\makeatother
\usepackage{xcolor}
\IfFileExists{xurl.sty}{\usepackage{xurl}}{} % add URL line breaks if available
\IfFileExists{bookmark.sty}{\usepackage{bookmark}}{\usepackage{hyperref}}
\hypersetup{
  hidelinks,
  pdfcreator={LaTeX via pandoc}}
\urlstyle{same} % disable monospaced font for URLs
\usepackage[margin=1in]{geometry}
\usepackage{graphicx}
\makeatletter
\def\maxwidth{\ifdim\Gin@nat@width>\linewidth\linewidth\else\Gin@nat@width\fi}
\def\maxheight{\ifdim\Gin@nat@height>\textheight\textheight\else\Gin@nat@height\fi}
\makeatother
% Scale images if necessary, so that they will not overflow the page
% margins by default, and it is still possible to overwrite the defaults
% using explicit options in \includegraphics[width, height, ...]{}
\setkeys{Gin}{width=\maxwidth,height=\maxheight,keepaspectratio}
% Set default figure placement to htbp
\makeatletter
\def\fps@figure{htbp}
\makeatother
\setlength{\emergencystretch}{3em} % prevent overfull lines
\providecommand{\tightlist}{%
  \setlength{\itemsep}{0pt}\setlength{\parskip}{0pt}}
\setcounter{secnumdepth}{-\maxdimen} % remove section numbering

\author{}
\date{\vspace{-2.5em}}

\begin{document}

\hypertarget{todo}{%
\section{TODO}\label{todo}}

Format with bioconductor vignette requirements.

\hypertarget{spatialutils}{%
\section{SpatialUtils}\label{spatialutils}}

R package with convenience functions for molecule-based spatial
transcriptomics data analysis (xenium by 10x, cosmx SMI by nanostring,
and merscope by vizgen). Introduces a new class to store ST data:
MoleculeExperiment class.

\hypertarget{why-the-moleculeexperiment-class}{%
\section{Why the MoleculeExperiment
class?}\label{why-the-moleculeexperiment-class}}

\begin{enumerate}
\def\labelenumi{\arabic{enumi})}
\tightlist
\item
  Enable easy analysis of spatial transcriptomics data at the molecule
  level, rather than the cell level.
\item
  Standardisation of molecule-based ST data across vendors, to hopefully
  facilitate comparison of different data sources.
\end{enumerate}

\hypertarget{the-me-object}{%
\section{The ME object}\label{the-me-object}}

\hypertarget{constructing-an-me-object}{%
\subsection{Constructing an ME object}\label{constructing-an-me-object}}

\begin{itemize}
\tightlist
\item
  instructions on how to use readMolecules() correctly
\item
  highlight benefits of how readMolecules() works e.g., readMolecules
  enables the user to decide if they want to keep all the data that is
  vendor-specific (e.g., qv in xenium).
\end{itemize}

\hypertarget{me-object-structure}{%
\subsection{ME object structure}\label{me-object-structure}}

\begin{itemize}
\item
  what are the slots? for now @molecules
\item
  what information does each slot contain? highlight how this enables
  standardisation of ST data across different vendors.
\item
  what is the format in which the information is stored? for now the
  list format
\item
  why is information stored like this? performance reasons? (e.g.,
  enabling minimal copying and thus less memory intensive?)
\end{itemize}

\hypertarget{methods-to-work-with-an-me-object}{%
\section{Methods to work with an ME
object}\label{methods-to-work-with-an-me-object}}

\begin{itemize}
\tightlist
\item
  Briefly introduce all methods that can be used to access and
  manipulate data in the @molecules slot.
\item
  Then explain each method in detail, and compare their performance when
  used on the ME implementation (list) or the naive implementation (df)
  of the data. Show microbenchmarking results in R code chunks

  \begin{itemize}
  \tightlist
  \item
    summarisation functions
  \item
    getters (data reading)
  \item
    setters (data manipulation)
  \item
    plotting functions
  \end{itemize}
\end{itemize}

\end{document}
